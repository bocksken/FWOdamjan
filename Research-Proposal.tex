%!TEX program = lualatex
\documentclass[11pt,dvipsnames,usenames,a4paper]{article}
\usepackage[UKenglish]{babel}
\usepackage{amssymb}
\usepackage{mathtools}
\usepackage{color}
\usepackage[left = 2.5 cm, right = 2.5cm, top = 3 cm, bottom = 3 cm]{geometry}
\usepackage{fontspec}
\setmainfont{Calibri}
\linespread{1.15}
\usepackage{framed,color}
\definecolor{shadecolor}{rgb}{0.85,0.85,0.85}
\usepackage{graphicx}
\usepackage{tabto}
\usepackage{enumitem}
\usepackage{fancyhdr}
\usepackage[super]{nth}
\usepackage[hang,bf,small]{caption}
\usepackage{float}
\usepackage{booktabs}
\usepackage{parskip}
\usepackage{hyperref}
\usepackage{cite}
% \usepackage{enumerate}

\let\OLDthebibliography\thebibliography
\renewcommand\thebibliography[1]{
	\OLDthebibliography{#1}
	\setlength{\parskip}{0pt}
	\setlength{\itemsep}{0.4pt plus 0.3ex}
}

\TabPositions{5cm, 10cm}

\begin{document}
%%%%%%%%%%%%%%%%%%%%%%%%%%%%%%
% Defining some commands
%%%%%%%%%%%%%%%%%%%%%%%%%%%%%%
\newcommand{\todo}[1]{\textcolor{red}{\emph{#1}}}

%%%%%%%%%%%%%%%%%%%%%%%%%%%%%%
% Personal information
%%%%%%%%%%%%%%%%%%%%%%%%%%%%%%
\pagenumbering{roman}

{\bf Date of the application} \tab\tab 1 April 2016 \\
{\bf Official internal deadline Ghent university} \tab 15 March 2016 \\
{\bf Unofficial internal deadline Ghent university} \tab {\bf\color{blue}24 March 2016} \\

%\vspace{2pt}

\begin{shaded}\centering GENERAL \end{shaded}
\textbf{Title of your research proposal} (\textcolor{Gray}{39/240 characters})\\
Imprecise continuous-time Markov chains\\[8pt]
\textbf{Dutch title} (\textcolor{Gray}{38/240 characters})\\
Imprecieze Markov-ketens in continue tijd\\[8pt]
\textbf{Summary in layman's terms}\\
\textcolor{Gray}{1469/1500 characters}\\
A Markov chain is a very popular probabilistic model, successful at describing the uncertain time evolution of various systems. 
It can be in any one of several states, and moves between them as time progresses. 
At each time and in each state, it is uncertain which state it will move to next, and this uncertainty is modelled using so-called transition probabilities, assumed to depend only on the current state. 

It turns out to be quite convenient to make predictions about the future behaviour of a Markov chain, and this makes it a popular tool in various applied domains, including engineering, artificial intelligence, mathematical finance and bio-informatics. 
But, to make these predictions reliable, perfect knowledge of the parameters---transition probabilities---of the Markov chain is required, and such knowledge is seldom available. 
%Consequently, the longer the process evolves, the less reliable the predictions of a Markov chain typically are. 
An efficient way of dealing with this problem, is to allow for partially specified parameters, using the theory of imprecise probabilities.

In the recent past, this idea has led to the successful development of so-called imprecise Markov chains in discrete time, and the universities of Ghent and Ljubljana have been at the forefront of this research. 
This project aims to formalise this mutual interest, and to use our combined expertise to develop a theory and toolbox for dealing with imprecise Markov chains in continuous time, making it possible to apply imprecise Markov chains in real-life problems, most of which involve continuous time.

% Since the majority of past research on imprecise Markov chains was conducted at the universities of Ghent and Ljubljana, project aims to formalize this long-term mutual interest, and 

% This project aims at generalizing the theory of Markov chains in continuous time, using imprecise probabilities, thereby extending the recent successful development of imprecise discrete time Markov chains to continuous time. The majority of the work on imprecise Markov models has been done at the Universities of Ghent and Ljubljana. We now intend to formalize this long time cooperation. 

% {\color{Gray}
% old example:

% Queueing theory is the study of waiting in line. It is often used to study problems in telecommunications, but also has applications in, for example, operations research. It studies systems which consist of servers, packets that need servicing and buffers to temporarily store these packets. The necessity of buffers is a consequence of the uncertainty in the arrival points and required service times of the packets.

% To model and reason with this uncertainty, we use methods from probability theory. These methods aid us in making smart design choices that enable us to achieve the desired system characteristics. One crucial problem is that most often we are not only uncertain about the arrival time points and service times themselves, but also about the validity of the probabilistic models we use for studying them. The theory of imprecise probability is a recent development of probability theory that is designed to deal with this so-called model uncertainty in a robust way.

% The project initially aims at further developing the part of this general theory that is relevant to continuous-time queueing theory. Afterwards, we will apply the developed methods and techniques to queueing applications and evaluate their usefulness. Expertise is needed from two research groups at Ghent University: SMACS, who are focused on queueing theory and applications in communication, and the Data Science Lab, whose expertise lies in robust uncertainty modelling using imprecise probabilities.
% }

\vspace{10pt}

\begin{shaded}\centering HOST INSTITUTION \end{shaded}
\textbf{Main host institution} \tab Ghent University \\
\textbf{Additional host institution(s)} \tab None\\
\textbf{Foreign host institution}\tab University of Ljubljana \\

\begin{shaded}\centering FUNDING PER HOST INSTITUTION \end{shaded}

% *** {\bf\color{blue}For each staff member, we must provide the following information:

% - Staff type (choose between fulltime scientist, parttime scientist, fulltime technician and parttime technician)

% - Motivation (in full text, max 1500 characters)

% - First Name

% - Last Name

% - Date of Birth (optional)

% - Academic degree (optional)

% - current employer (optional)

% - a required amount per year (for 2017, 2018, 2019 and 2020 separately) (the amount in the first year should be equal or higher than during the other years)}

{\bf GHENT UNIVERSITY}

{\it STAFF}

{\bf First PhD student Ghent}\\
Staff type: fulltime scientist\\
Motivation: (in full text, max 1500 characters)\\
First Name: as yet unknown\\
Last Name: as yet unknown\\
Required amount per year:\\
2017: 45000\\
2018: 45000\\
2019: 45000\\
2020: 45000


{\bf Second PhD student Ghent}\\
Staff type: fulltime scientist\\
Motivation: (in full text, max 1500 characters)\\
First Name: as yet unknown\\
Last Name: as yet unknown\\
Required amount per year:\\
2017: 45000\\
2018: 45000\\
2019: 45000\\
2020: 45000

{\it CONSUMABLES}

{\bf\color{blue}For each off the consumables we request, we must provide the following information:

- Consumable type (choose between Other, Publication costs, Research expenses, Small equipment < euro 20000, Travel and accomodation costs)

- Detailed description of consumables (in full text, max 1500 characters)

- Motivation (in full text, max 1500 characters)

- a required amount per year (for 2017, 2018, 2019 and 2020 separately) (the amount in the first year should be equal or higher than during the other years)
}

{\bf ???}\\
Consumable type: Other\\
Detailed description of consumables (in full text, max 1500 characters): *** ??? ***\\
Motivation (in full text, max 1500 characters): *** ??? ***\\
Required amount per year:\\
2017: *** ??? ***\\
2018: *** ??? ***\\
2019: *** ??? ***\\
2020: *** ??? ***


{\bf UNIVERSITY OF LJUBLJANA}

{\it STAFF}

\textbf{PhD student Ljubljana} \\
Staff type: fulltime scientist\\
Motivation: (in full text, max 1500 characters)\\
First Name: as yet unknown\\
Last Name: as yet unknown\\
Required amount per year:\\
2017: 30708\\
2018: 30708\\
2019: 30708\\
2020: 30708

\textbf{Parttime researcher Ljubljana}\\
Staff type: fulltime scientist\\
Motivation: Matja\v{z} Omladi\v{c} is an experienced researcher in mathematics, with expertise in many fields, including linear algebra, operator theory, probability and statistics. His contribution to this project will include cooperation on computational methods and limit behaviour.\\
First Name: Matja\v{z}\\
Last Name: Omladi\v{c}\\
Date of Birth (optional): 8 August 1950\\
Academic degree (optional): Full Professor\\
Current employer (optional): University of Ljubljana\\
Required amount per year:\\
2017: 10235\\
2018: 10235\\
2019: 10235\\
2020: 10235

% \textbf{Matjaž Omladič:} \\ 
% part-time scientist

% 2017: \\
 
% Salaries: EUR 8818 \\
% Contributions of employer: EUR 1417 \\
% {\bf Total: EUR 10235} \\

% 2018: \\
% Salaries: EUR 8818 \\
% Contributions of employer: EUR 1417 \\
% {\bf Total: EUR 10235} \\
 

% 2019: \\
% Salaries: EUR 8818 \\
% Contributions of employer: EUR 1417 \\
% {\bf Total: EUR 10235} \\

% 2020: \\

% Salaries: EUR 8818 \\
% Contributions of employer: EUR 1417 \\
% {\bf Total: EUR 10235} \\


% full-time scientist

% 2017: \\

% Salaries: EUR 26454 \\
% Contributions of employer: EUR 4253 \\
% {\bf Total: EUR 30708} \\

% 2018: \\

% Salaries: EUR 26454 \\
% Contributions of employer: EUR 4253 \\
% {\bf Total: EUR 30708} \\

% 2019: \\

% Salaries: EUR 26454 \\
% Contributions of employer: EUR 4253 \\
% {\bf Total: EUR 30708} \\

% 2020: \\

% Salaries: EUR 26454 \\
% Contributions of employer: EUR 4253 \\
% {\bf Total: EUR 30708} \\

{\it CONSUMABLES}

{\bf\color{blue}For each off the consumables we request, we must provide the following information:

- Consumable type (choose between Other, Publication costs, Research expenses, Small equipment < euro 20000, Travel and accomodation costs)

- Detailed description of consumables (in full text, max 1500 characters)

- Motivation (in full text, max 1500 characters)

- a required amount per year (for 2017, 2018, 2019 and 2020 separately) (the amount in the first year should be equal or higher than during the other years)
}


*** The Slovene budget follows the rules of Slovene Research Agency (ARRS) that are prescribed for research project, with the chosen C category for evaluation of research costs, and follows the distributions of costs in types as specified by the ARRS ***


2017: \\
Other (Administrative costs: faculty overhead: 14.5 \% deducted by our faculty for administrative costs)): EUR 10663 \\
Small equipment (computer. eq, software, etc.): EUR 9000 \\
Research expenses: (printing, copying, books): EUR 1500 \\
Travel and accommodation costs (meetings of research partners, attendance to conferences): EUR 11000  \\
Other costs: EUR 435 \\
\textbf{Total: EUR 32598} \\

2018: \\
Other (Administrative costs: faculty overhead: 14.5 \% deducted by our faculty for administrative costs)): EUR 10663 \\
Small equipment (computer. eq, software, etc.): EUR 6000 \\
Research expenses: (printing, copying, books): EUR 1500 \\
Travel and accommodation costs (meetings of research partners, attendance to conferences): EUR 14000  \\
Other costs: EUR 435 \\
\textbf{Total: EUR 32598} \\

2019: \\
Other (Administrative costs: faculty overhead: 14.5 \% deducted by our faculty for administrative costs)): EUR 10663 \\
Small equipment (computer. eq, software, etc.): EUR 6000 \\
Research expenses: (printing, copying, books): EUR 1500 \\
Travel and accommodation costs (meetings of research partners, attendance to conferences): EUR 14000  \\
Other costs: EUR 435 \\
\textbf{Total: EUR 32598} \\

2020: \\
Other (Administrative costs: faculty overhead: 14.5 \% deducted by our faculty for administrative costs)): EUR 10663 \\
Small equipment (computer. eq, software, etc.): EUR 6000 \\
Research expenses: (printing, copying, books): EUR 1500 \\
Travel and accommodation costs (meetings of research partners, attendance to conferences): EUR 14000  \\
Other costs: EUR 435 \\
\textbf{Total: EUR 32598} \\


*** {\bf\color{blue} you can also ask for equipment, but that is for large and very costly equipment, which does not apply to us} ***

!! The FWO rules only apply to the Belgian part of the funding.

We also have to specify the Slovenian funding in detail, but the FWO will only check basic rules for that funding, such as that the maximum cannot exceed the Slovenian limit, and that the Belgian part of the funding should be larger than the Slovenian part. The details of the Slovenian funding will be checked by the Slovenian agency, which means that if they are OK with using the funding to partially pay people, then the FWO does not care about that (except maybe for the supervisor, I'm not yet 100\% sure about that). Similarly for putting the same Slovenian person on multiple projects: if the Slovenian agency agrees, then it should be OK from an administrative point of view. However, I do think that we should be careful with this. If the jury notices that the same person intends to work (get payed from) two different projects, perhaps this has a negative effect on our chances.

{\color{Gray}These are the rules (for the Belgian part of the funding):

You are not allowed to request funding for foreign institutions or institutions belonging to the French-speaking community of Belgium. The staff and consumables that are applied for in the first year, have to be equal or higher compared with the funds asked for during the other years. 

For each project, and in case of an interuniversitairy project, for each host institution, one can apply for €45.000 till €130.000 each year including research staff and consumables. In case one of the project partners only requests funding for consumables, the lower limit for this partner is set at €20.000. 

The real cost is used when the name(s) of the researcher(s) is (are) already known. When the name(s) of the researcher(s) is (are) not yet known, the following amounts may be used as indicative costs:

·         Bursary: €45,000

·         Scientific staff, 0 years of seniority: €65,000

·         Postdoc researcher, 4 years of seniority: €85,000

·         Technical staff, 6 years of seniority: €50,000


Additionally, it is possible to request funds for equipment up to €150.000. Matching funding is allowed up to €150.000.
}


{\bf\color{blue}*** We need to include a (standard) CV for the personnel that will be appointed on the project (if we know them already). Furthermore, for a each of the (co-)supervisors, we need a brief one-page CV (The first half page should consist of full text, and should difinitely contain the current position, other appointments relevant for this application, previous scientific awards received and other relevant information to evaluate the scientific CV. The second half of the page should list 5 key peer reviewed publications that are representative for the (co-)promotor's scientific career. ***}

{\color{Gray}
These are the rules:

Provide a short CV of the personnel to be appointed on this project and already involved. 

Also include a short CV (max. 1 page) for both the Flemish and foreign supervisor and all co-supervisors with 5 key peer reviewed publications that are representative for his/her scientific career. You can provide this in full text taking into account the current position and other appointments relevant for this application. It can also be important to mention previous scientific awards received and other relevant information to evaluate the scientific CV.}\\

\begin{shaded}\centering SUPERVISORS \end{shaded}
\vspace{2pt}

\textbf{Supervisor:} Gert de Cooman\\
Title: Prof.\\
First name: Gert\\
Surname: De Cooman\\
Date of birth: 30/09/1964\\
Current occupation: Professor\\
Employment (\%): 100\\
E-mail: gert.decooman@ugent.be\\
Institution: Ghent University\\
Research unit: Imprecise Probability unit of the Data Science Lab\\
Street and number: Technologiepark-Zwijnaarde 914\\
Postal Code: 9052\\
City: Zwijnaarde (Gent)\\
Country: Belgium\\[7pt]
\textbf{Co-supervisor:} Jasper De Bock\\
Title: Dr.\\
First name: Jasper\\
Surname: De Bock\\
Date of birth: 04/09/1988\\
Occupation: Postdoctoral FWO Fellow\\
Employment (\%): 100\\
E-mail: jasper.debock@ugent.be\\
Institution: Ghent University\\
Research unit: Imprecise Probability unit of the Data Science Lab\\
Street and number: Technologiepark-Zwijnaarde 914\\
Postal Code: 9052\\
City: Zwijnaarde (Gent)\\
Country: Belgium\\[7pt]
\textbf{Foreign supervisor:} Damjan {\v S}kulj\\
Title: Prof.\\
First name: Damjan\\
Surname: {\v S}kulj\\
Date of birth: {11/03/1975}\\
Current occupation: {Associate professor}\\
Employment (\%): {100}\\
E-mail: {damjan.skulj@fdv.uni-lj.si}\\
Institution: University of Ljubljana\\
Research unit: {Faculty of Social Sciences}\\
Street and number: {Kardeljeva plo\v s\v cad 5}\\
Postal Code: {SI-1000}\\
City: {Ljubljana}\\
Country: {Slovenia}\\%[5pt]


%\textbf{Foreign co-supervisor} \tab ???\\

%\vspace{5pt}

% {\color{Gray}
% These are the rules:

% A research project is executed under the direction of supervisors in cooperation with one or more co-supervisors. The supervisors and co-supervisors need to comply with the conditions stipulated in article 9 of the rules for research projects.

% The (co-)supervisors are the actual initiators of the project, and as such are responsible for it. The foreign supervisor and co-supervisors accept that the supervisor, appointed at a Flemish host institution, will act as supervisor-spokesperson towards FWO. {\bf It is not possible to use the available budget to finance the (co-)supervisors ’salaries.}

% A researcher can only act as a (co-)supervisor for maximum two projects per application round.
% }

{\color{blue}\bf *** A detailed list of all publications of all the (co-)supervisors should be submitted through the FWO E-portal before the final submission date of the call!
NOTE: These publications do not have to be sent as a separate attachment nor as part of another attachment, but should be entered into the E-loket of the FWO. ***}

\vspace{5mm}

\begin{shaded}\centering ETHICS \end{shaded}

Not relevant

\vspace{5mm}

\begin{shaded}\centering DISCIPLINES \end{shaded}

{\bf Scientific field} \tab Science and Technology \\
{\bf FWO Expert Panel} \tab Informatics and Knowledge Technology (W\&T5) \\[8pt]
{\bf Motivation of panel choice}\\[6pt]
This project aims to develop a theory of imprecise continuous-time Markov chains. On the one hand, this topic finds applications in such fields as queueing theory, which abound in computer and telecommunication networks. On the other hand, this topic falls squarely within the field of imprecise probabilities. Probabilistic approaches have always been prominent in artificial intelligence and data mining, and recently, the more specific case of imprecise probabilities has been well received at various AI conferences (best student paper award at ECSQARU 2013 and UAI 2013 and best paper award at ECSQARU 2013). For these reasons, the expert panel should be eminently able to judge the merits and impact of this research proposal.\\[8pt]
{\bf Disciplines}\\[6pt]
We need to choose (at most five) disciplines from some list.

I would choose the following:\\[5pt]
- Probability theory;\\
- Knowledge representation and reasoning.

There were also some other relevant disciplines in the list (Artificial intelligence, Differential equations and mathematical physics, Functional analysis, Financial mathematics, Mathematical modelling, Algorithms and computational mathematics, Statistics) but I think it would not be tactical to include them, since they are mostly mathematical, and we are applying to the computer science panel.


\vspace{5mm}

\begin{shaded}\centering REFEREES \end{shaded}

\vspace{5pt}
% Damjan's suggestions:
% \vspace{5pt}

{\bf Lev Utkin}\\
Title: Prof.\\
First name: Lev\\
Surname:  Utkin \\
Current occupation: Professor\\
E-mail: lev.utkin@list.ru\\
Institution: Peter the Great St.Petersburg Polytechnic University\\
Research unit: Telematics Department\\
Street and number: Polytechnicheskaya, 29\\
Postal Code: 195251\\
City: St.Petersburg\\
Country: Russia\\[-7pt]

{\bf Paolo Vicig}\\
Title: Prof.\\
First name: Paolo\\
Surname:  Vicig \\
Current occupation: Professor\\
E-mail: PAOLO.VICIG@deams.units.it\\
Institution: University of Trieste\\
Research unit: Dipartimento di Scienze Economiche, Aziendali, Matematiche e Statistiche\\
Street and number: Piazzale Europa 1\\
Postal Code: 34127\\
City: Trieste\\
Country: Italy\\[-7pt]

% {\bf Renato Pelessoni}\\
% Title: Prof.\\
% First name: Renato\\
% Surname:   Pelessoni\\
% Current occupation: Associate Professor\\
% E-mail: RENATO.PELESSONI@deams.units.it\\
% Institution: University of Trieste\\
% Research unit: Dipartimento di Scienze Economiche, Aziendali, Matematiche e Statistiche\\
% Street and number: Piazzale Europa 1\\
% Postal Code: 34127\\
% City: Trieste\\
% Country: Italy\\[-7pt]

% {\bf Andrea Wiencierz}\\
% Title: Dr.\\
% First name: Andrea\\
% Surname:  Wiencierz\\
% Current occupation: Lecturer\\
% E-mail: andrea.wiencierz@york.ac.uk\\
% Institution: University of York\\
% Research unit: Department of Mathematics\\
% Postal Code: YO10 5DD\\
% City: York\\
% Country: United Kingdom\\[-7pt]

% {\bf Marco Cattaneo}\\
% Title: Dr.\\
% First name: Marco\\
% Surname:  Cattaneo\\
% Current occupation: Lecturer in Mathematics\\
% E-mail: m.cattaneo@hull.ac.uk\\
% Institution: University of Hull\\
% Research unit: Department of Physics and Mathematics\\
% Street and number: Robert Blackburn Building, 066B\\
% Postal Code: HU6 7RX\\
% City: Hull\\
% Country: United Kingdom\\[-7pt]

{\bf Alessio Benavoli}\\
Title: Dr.\\
First name: Alessio\\
Surname:  Benavoli\\
Current occupation: Senior Researcher \\
E-mail: alessio@idsia.ch\\
Institution: Studi sull'Intelligenza Artificiale (IDSIA)\\
Research unit: Istituto ``Dalle Molle'' di 
Studi sull'Intelligenza Artificiale (IDSIA )\\
Street and number: Galleria 2\\
Postal Code: CH-6928\\
City: Manno (Lugano)\\
Country: Switzerland\\[-7pt]

% {\bf Fabio Cuzzolin}\\
% Title: Dr.\\
% First name: Fabio\\
% Surname:  Cuzzolin\\
% Current occupation: Reader in Computer Vision\\
% E-mail: fabio.cuzzolin@brookes.ac.uk\\
% Institution:  Oxford Brookes University\\
% Research unit: Department of Computing and Communication Technologies \\
% Street and number: Wheatley Campus\\
% Postal Code: OX33 1HX\\
% City: Wheatley (Oxford)\\
% Country: United Kingdom\\[-7pt]


{\bf Glenn Shafer}\\
Title: Prof.\\
First name: Glenn\\
Surname:  Shafer \\
Current occupation: Professor\\
E-mail: gshafer@business.rutgers.edu
\\
Institution: Rutgers Business School Newark and New Brunswick\\
Research unit: Department of Accounting and Information Systems\\
Street and number: 1 Washington Park\\
Postal Code: 07102\\
City: Newark (New Jersey)\\
Country: United States of America\\[-7pt]

{\bf Vladimir Vovk}\\
Title: Prof.\\
First name: Vladimir\\
Surname:  Vovk\\
Current occupation: Professor\\
E-mail: v.vovk@rhul.ac.uk\\
Institution: Royal Holloway, University of London\\
Research unit: Computer Learning Research Centre, Department of Computer Science\\
Street and number: Egham Hill\\
Postal Code: TW20 0EX\\
City: Egham (Surrey)\\
Country: United Kingdom\\[-7pt]

{\bf Philip Dawid}\\
Title: Prof.\\
First name: Philip\\
Surname:  Dawid\\
Current occupation: Emeritus Professor of Statistics\\
E-mail:  apd25@cam.ac.uk\\
Institution: University of Cambridge\\
Research unit: Centre for Mathematical Sciences\\
Street and number: Wilberforce Road\\
Postal Code: CB3 0WB\\
City: Cambridge\\
Country: United Kingdom\\[-7pt]

{\bf Teddy Seidenfeld}\\
Title: Prof.\\
First name: Teddy\\
Surname: Seidenfeld \\
Current occupation: Professor of Philosophy and Statistics\\
E-mail: teddy@stat.cmu.edu\\
Institution: Carnegie Mellon University\\
Research unit: Department of Philosophy\\
Street and number: 5000 Forbes Avenue\\
Postal Code: PA 15213\\
City:  Pittsburgh\\
Country: United States of America\\[-7pt]

{\bf Didier Dubois}\\
Title: Prof.\\
First name: Didier\\
Surname:  Dubois \\
Current occupation: Research Advisor\\
E-mail: dubois@irit.fr\\
Institution: Universit\'e Paul Sabatier\\
Research unit: l'Equipe``Raisonnements Plausibles, Décision, et Méthodes de Preuves''\\
Street and number: 118 Route de Narbonne\\
Postal Code: 31062 CEDEX 9\\
City: Toulouse\\
Country: France\\[-7pt]

{\bf Linda van der Gaag}\\
Title: Prof.\\
First name: Linda\\
Surname:  van der Gaag\\
Current occupation: Professor\\
E-mail: L.C.vanderGaag@uu.nl\\
Institution: Universiteit Utrecht\\
Research unit: Department of Information and Computing Sciences, Decision Support Systems group\\
Street and number: Princetonplein 5, De Uithof\\
Postal Code: 3584 CC\\
City: Utrecht\\
Country: The Netherlands\\[-7pt]

{\bf Seraf\'in Moral}\\
Title: Prof.\\
First name: Seraf\'in\\
Surname:  Moral \\
Current occupation: Lecturer in Computer Sciences and Artificial Intelligence\\
E-mail: smc@decsai.ugr.es\\
Institution: University of Granada\\
Research unit: Department of Computer Science and Artificial Intelligence, the Uncertainty in Artificial Intelligence research group\\
Street and number: C// Periodista Daniel Saucedo Aranda, s/n\\
Postal Code: 18071\\
City: Granada\\
Country: Spain\\[-7pt]

% {\bf Fabio Cozman}\\
% Title: Prof.\\
% First name: Fabio\\
% Surname:  Cozman\\
% Current occupation: *** ??? ***\\
% E-mail: *** ??? ***\\
% Institution: *** ??? ***\\
% Research unit: *** ??? ***\\
% Street and number: *** ??? ***\\
% Postal Code: *** ??? ***\\
% City: *** ??? ***\\
% Country: *** ??? ***\\[-7pt]


% \vspace{10pt}
% *** Please list 10 potential referees below and provide contact details.
% The FWO administration will contact referees from this list in a random way. ***

% *** For each of the referees, we need to provide the same administrative info as for the supervisors (see the supervisors section) ***

% {\color{Gray}
% These are the rules:

% Referees should be appointed at an university, research institution or research entity of another type of organization and at least at postdoctoral level.

% Not eligible as referee are:
% members of the Board of Directors of the FWO;
% members of an FWO expert panel;
% persons appointed to a Belgian university, research institute or any other organization or, in the case of calls for proposals in the framework of bilateral or lead agency agreements, persons appointed to similar institutions or organisations in the country where the foreign project partner is professionally active;
% persons with a professional appointment to a foreign institute where the applicant(s) has been enrolled as a student or professional after January 1st of the year n-3 (n=year of application);
% any co-authors with the applicants of a publication that was submitted or published after January 1st of the year n-3 (n=year of application); 

% ‘Co-authorship’ is to be understood as follows:
% co-authorship of a monography of which the applicant is co-author as well;
% co-authorship of an article or another type of contribution to a collection (book, journal issue, report, congress proceedings, abstract, …) of which the applicant is co-author as well.
% Editors are not regarded as co-authors insofar as they have not also acted as what is understood under ‘co-author’ as described above. Co-editors of the applicant are not accepted as an external referee.
% partners of the applicant(s) in a research cooperation, whether formalised in a research project or not, that has been applied for or has been running after January 1st of the year n-3 (n=year of application. In this context, the following shall in any case qualify as research cooperation (non-exhaustive list):
% Cooperation under a research fellowship, granted by the FWO;
% Cooperation under a research project, whether relating to a specific subject or not or under an international cooperation project, granted by the FWO;
% Cooperation under the Odysseus programme or the Big Science programme, granted by the FWO;
% Cooperation under a Scientific Research Network, granted by the FWO;
% Cooperation under programmes similar to those mentioned above, granted by organisations other than the FWO;
% Joint research work not formalised in a cooperation structure as defined above;
% Research carried out in the research areas and/or with research facilities provided by the applicant to the referee or vice versa;
% ...
% The applicants are responsible for the eligibility of the proposed referees. Whenever the proposed referees do not comply with the eligibility criteria, the application will be declared ineligible.

% In case the applicant(s) doubts the eligibility of one or more of the proposed referees, he or she can also contact the FWO through his/her e-loket account before submitting the application. The questions concerning eligibility that reached the FWO before the application was submitted will be presented to the FWO referee commission of the appropriate scientific domain, consisting of all expert panels’ chairs of that domain. Five referee commissions are established, one for each domain: biological sciences, humanities, social sciences, medical sciences and science \& technology; for applications submitted to the Interdisciplinary Panel the referee commissions of the respective scientific domains will be consulted. In case co-authorship is detected in publications with ten or more authors, the FWO administration will consult the referee commission as well. In all the above cases, the referee commission will decide on the alleged eligibility of the proposed referees. When the referee commission decides negatively on the eligibility of a proposed referee in an application that has already been submitted, this application will be disqualified. When the referee commission answers negatively to a question concerning the eligibility of a proposed referee that reached the FWO before the application was submitted, the applicant will be asked to propose a new referee that does meet the eligibility criteria. 
% After the administrative check, the FWO will inform the applicant about the violations that were found. In case the alleged violations result from a factual error of the FWO administration, the FWO can be notified. 
% For the integral regulations on internal and external peer review, see: http://www.fwo.be/en/the-fwo/organisation/fwo-expertpanels/regulations-fwo–-internal-and-external-peer-review/.}

\vspace{10pt}

\begin{shaded}\centering EXTRA DATA \end{shaded}

\textbf{Mention other funding, applied for elsewhere or already available. (Optional)}\\
\textcolor{Gray}{0/3000}\\

% {\color{blue}\bf This part should only mention funding that is related to the same project proposal} 

% I would suggest to leave it empty...

% *** should we mention the projects of Stavros and Alexander? ***

% @Damjan: do you have stuff that should be mentioned here?

% I am a member of the research group: Social Sciences Methodology, Statistics and Informatics; 
% Code	P5-0168 (B); Period	1.1.2015 - 31.12.2020; Head:	Ferligoj Anuška

\vspace{7pt}




\vspace{5mm}

\begin{shaded}\centering RESEARCH CONTEXT \end{shaded}
\textbf{What is the added value of this scientific collaboration.}\\
\textit{Elaborate on the complementary expertise of the project partners and explain how the project parts are integrated and relevant for the scientific input from both sides. Explain how this project fits in the research activities of your research group and the foreign research group. If the project has already been initiated, please state the progression of your research.}\\
\textcolor{Gray}{2951/3000 characters}

At Ghent University, the research for this project will be conducted by the Imprecise Probability unit of Data Science Lab, which has world-leading expertise in basic and applied research on imprecise probabilities.
The group is led by Gert de Cooman, who is Full Professor in Uncertainty Modelling and Systems Science, and Honorary Visiting Professor at Durham University.
It currently consists of two post-docs (Jasper De Bock and Erik Quaeghebeur) and four PhD students, who work in a wide range of areas in imprecise probabilities, including credal networks, discrete-time imprecise Markov chains and queueing, and the foundations of statistical and probabilistic inference.

At the University of Ljubljana, research for this project will be led by Damjan \v{S}kulj, who is an associate professor of Applied Mathematics. 
He has many years of experience in working with imprecise probabilities, and over the last couple of years, he has published several articles on discrete-time imprecise Markov chains. 
Recently, he has published a preliminary study of imprecise continuous-time Markov chains, which has served as the basis for the line of research that we propose here.
Within the context of this project, he aims to build his own research group, which will initially consist of a new PhD student, and prof.~dr.~Matja\v{z} Omladi\v{c}, who is an experienced researcher with expertise in many fields of mathematics, including linear algebra, probability theory and statistics.

The reason why the two groups are joining forces for this project, is that they have both been actively involved in the development of imprecise discrete-time Markov chains, to the extent that they are now recognised as the two main authorities on the topic. 
Clearly, the skills that these groups have developed during their previous---independent---studies of imprecise discrete-time Markov chains will be a major asset for the project, which aims to develop a theory of \emph{continuous-time} Markov chains. 
Even more so because both groups tend to approach the topic of imprecision in Markov chains from partly different and complementary angles.

The focus of the UGhent group lies more on developing the basic theory of imprecise Markov chains within the context of more general stochastic processes, and to place this theory on solid mathematical grounds. 
The focus of the Ljubljana group is more numerically and computationally oriented. 
Both approaches, however, are interlinked and dependent on each other. 
On the one hand, given the complexity of imprecise probability models, theoretical results will be required in order to be able to develop efficient computational methods. 
On the other hand, again due to the inherent complexity of imprecise models, theoretical problems that can traditionally be solved analytically---such as finding limit distributions---will now require the use of efficient computational methods. 
Therefore, we expect our respective approaches to be mutually beneficial.

%As is the case for any theory that uses imprecise probabilities, the aim is to develop methods that enable reasoning with partially known parameters, which consistently produces models that are of much greater complexity than their precise counterparts. It is therefore important that their interpretation remains within the spirit of the general theory of probability and at the same time the complexity must be reasonable in order to be useful and applicable in practice.



\textbf{Provide the national and international context of the project.}\\
\textit{Mention research collaborations, larger projects, programmes and international networks in which your research can be situated.}\\
\textcolor{Gray}{1499/1800 characters}

The two research partners for the project both have strong ties with other leading researchers in imprecise probabilities, and their members have been actively involved in SIPTA (Society for Imprecise Probability: Theories and Applications), which is an umbrella organisation (and statistical society) that joins researchers on imprecise probabilities from different groups all over the world and promotes their work. 
On the one hand, members of both groups have been involved in the organisation of several events for this society, including the series of biennial ISIPTA conferences (International Symposia on Imprecise Probability: Theories and Applications) and SIPTA summer schools, and the series of WPMSIIP workshops (Workshop on Principles and Methods of Statistical Inference with Interval Probability). 
On the other hand, the supervisor of this project (Gert de Cooman) is a founding member and former president of this society, and the co-supervisor (Jasper De Bock) is a current at-large member of its executive committee. 
The partners in this project also have ties with other leading research groups in imprecise probabilities, at Carnegie Mellon University (US), IDSIA (Switzerland), Durham University (UK), Universidade de S\~ao Paulo (Brazil), IRIT (France), Ludwig-Maximilians-Universit\"at M\"unchen (Germany), Universidad de Oviedo and Universidad de Granada (Spain), Universiteit Utrecht (The Netherlands), and University of Trieste (Italy). 

\textbf{Describe the past cooperation between the project partners.}\\
\textcolor{Gray}{1008/1800 characters}

The two partners for this project were both invited to contribute to the book ``Introduction to Imprecise Probabilities'' (Wiley, 2014). 
In particular, as a consequence of their mutual interest and expertise in imprecise Markov chains, Damjan \v{S}kulj (the foreign supervisor of this project) and Filip Hermans (a former member of the Ghent group) jointly wrote the ``Stochastic processes'' chapter, in which they introduce a general theory of imprecise stochastic processes, with an emphasis on the theory of discrete-time imprecise Markov chains.

Other forms of cooperation between the project partners were of a more informal nature. 
In the context of their common participation in---and their membership in the programming committees of---the series of ISIPTA conferences, the members of both groups have met on multiple occasions, and they have used these opportunities to discuss their work on imprecise Markov chains. 
Such informal discussions lie at the basis of this joint project proposal.

%%%%%%%%%%%%%%%%%%%%%%%%%%%%%%
% The actual research proposal
%%%%%%%%%%%%%%%%%%%%%%%%%%%%%%
\newpage
\pagenumbering{arabic}

\setcounter{page}{1}

\begin{shaded}\centering PROJECT OUTLINE \end{shaded}


\textbf{Indicate the state of the art.}

This project is concerned with imprecise Markov chains: robust generalisations of Markov chains, based on the theory of imprecise probabilities. 
In this state of the art section, we briefly introduce Markov chains and imprecise probabilities, explain how they can be combined to obtain imprecise Markov chains, and then go on to give an overview of recent advances in this field.

% \vspace{5pt}
\emph{Markov chains}\\[5pt]
A \emph{discrete-time} \emph{finite-state} \emph{Markov chain} is a special stochastic process. 
It models the time evolution in discrete time steps of a system that can be in a finite number of states.
It is \emph{stochastic} because the model is given in terms of probabilities, such as, for example, the probability to be in a certain state at a given time.
It is a Markov chain because the probabilistic model satisfies a \emph{Markov condition}: the (so-called) transition probability to be in a certain state at the next time step only depends on the current state, and not on the past states.
The probabilistic time evolution of the state of such Markov chains is described by linear difference equations.
In \emph{continuous-time Markov chains}, the time steps become infinitesimally small, and their time evolution is described by linear differential equations. 
Both discrete and continuous-time Markov chains belong to the most powerful, and---due to their reasonable computational complexity---the most widely used probabilistic models in a very wide range of applications in engineering (filtering, control, queueing), AI (text and speech recognition), mathematical finance and bio-informatics, to name only a few domains.

% \vspace{5pt}
\emph{Imprecise probabilities and the limitations of precise models}\\[5pt]
Nevertheless, these Markov chains have limitations: we will focus here on two of them.
The first is that the uncertainty about the state is described by (transition) probabilities: real numbers whose precision is almost always unwarranted in applications, because of the limited accuracy and reliability of statistical estimation methods, and the essentially limited numerical accuracy that computer simulations offer.
The second limitation is related to the Markov condition: that the probabilistic model for the next state depends only on the current one, is quite a strong assumption to make. 
It leads to a drastic decrease in computational complexity, and is therefore often made, but is typically hard to justify in practical applications.
The justification that is commonly given for both of these assumptions in modelling contexts is that the simplification they entail is unlikely to have major implications for the conclusions that we draw from the models.
However, this claim is hard to verify, and often unwarranted.
As a brief look at the mathematics of these stochastic models will already show, there are many situations where (i) the conclusions of a Markov chain analysis depend heavily and crucially on the precise values of the transition probabilities, or where (ii) not making a Markov assumption leads to qualitatively very different results.

Interestingly, in recent decades much progress has been made in extending the existing corpus of general probabilistic knowledge to deal with both types of limitations: the field of \emph{imprecise probabilities} \cite{walley1991,augustin2013:itip,troffaes2013:lp} has, amongst other things, developed well-justified, mathematically rigorous, as well as robust and efficient methods for dealing with both of them.
Stated in very simple terms, an imprecise probability model is a probabilistic model that is partially specified.
For example, whenever it is infeasible to reliably estimate the probability of some event, the theory imprecise probabilities allows for the use of a probability interval instead.
Such partial specifications do not lead to a unique probability measure, but instead give rise to a set of compatible probability measures.
These sets of probability measures are called credal sets.
They are the basic uncertainty models in imprecise probability theory. 
A credal set can be represented equivalently by a so-called \emph{lower expectation}: a non-linear operator that is the lower envelope of the linear expectation operators associated with the probability measures in the credal set.
Remarkably, the state of the art in the theory is able to perform the necessary robust calculations and derivations for all of the (infinite number of) precise probability measures in such a credal set, with a computational efficiency that is not much lower than---and often as low as---the one for the precise models; see for instance \cite{cooman2009,debock2014:estihmm,debock2015:thesis,cooman2008} for concrete examples that substantiate this claim.

% \vspace{5pt}
\emph{A brief introduction to imprecise discrete-time Markov chains}\\[5pt]
Applying the results and ideas developed in the field of imprecise probabilities has led to interesting advances in \emph{discrete-time} finite-state Markov chains, resulting in a theory of so-called \emph{imprecise Markov chains} \cite{cooman2008,hermans2012,hartfiel1998,skulj2013,cooman2015:markovergodic}.
They correspond to a collection of stochastic processes that are only `superficially Markov', in the sense that their \emph{sets} of transition probabilities satisfy a Markov condition, whereas the individual members of those sets need not.
Imprecise Markov chains are \emph{not} simply collections of precise Markov chains, but rather correspond to collections of general stochastic processes whose transition models belong to sets that satisfy a Markov condition.
In this sense, they lead to more robust and reliable inferences than their precise counterparts, and nevertheless still allow for very efficient and elegant computations, a combination that is clearly important in applications.

\emph{Imprecise probability trees and submartingales}\\[5pt]
A crucial step that made these developments possible was the observation that general discrete-time stochastic processes can be described elegantly using probability trees: graphs whose \emph{paths} from root to leaves represent possible time evolutions of the process, whose nodes (called \emph{situations}) represent time evolutions up to a certain time step, and where each situation has a local probability model for what may happen at the next time step.
The paths correspond to the elements of the sample space in the more conventional measure-theoretic approach.
With such a tree there is associated a convex closed set of so-called martingales---real processes with zero expected increase from one time step to the next---and an important result by Ville \cite{ville1939,shafer2001} shows that all probabilistic properties of, and inferences about, the stochastic process are completely determined by this set of martingales. Going to imprecise stochastic processes now becomes fairly straightforward, at least conceptually: the local models in the tree's nodes are now sets of probabilities, and all probabilistic inferences are, through an extension of Ville's Theorem, characterised by the corresponding convex closed set of \emph{sub}martingales (real processes with non-negative expected increase under all precise probability models in these sets) \cite{cooman2007d,cooman2015:markovergodic}.

\emph{State of the art for discrete-time imprecise Markov chains}\\[5pt]
There have been several approaches to extending Markov chain theory to deal with imprecision. 
Hartfiel \cite{hartfiel1998}, using the name \emph{Markov set-chains}, focussed on lower and upper bounds for the transition probabilities between individual states, and also discussed sets of transition operators, recognising the importance of convexity of sets of distributions over states, separately specified rows and coefficients of ergodicity. 
Unaware of Hartfiel's work, Kozine and Utkin~\cite{utkin:02} proposed an approach with a different underlying interpretation, and very different computational aspects. 
Instead of allowing transition probabilities to vary within given bounds from one time step (or state) to the next, they assumed a single precise but unknown Markov model: they worked with sets of precise Markov chains. 

The first systematic anf fully rigorous approach to combining discrete-time Markov chains with imprecise probabilities is due to De Cooman et al.~\cite{cooman2008}, using the idea of lower conditional expectations---also called lower transition operators. 
Similar ideas were explored independently by Škulj~\cite{skulj:09}, who focussed instead on interval probability models and their associated convex sets of probability distributions. 
For calculations and inferences, the main difference between the two approaches is that the lower transition operator approach in \cite{cooman2008} allows for very efficient backwards induction, in contrast with the much less efficient forward calculations typical in the sets of distributions approaches \cite{skulj:09}. 
Lower transition operators, and the ideas used in the analysis of the limit behaviour of imprecise Markov chains, introduced in \cite{cooman2008}, became the basis for most of the later work on imprecise Markov chains. 
They also emerge naturally as special cases in the above-mentioned theory of imprecise stochastic processes based on imprecise probability trees and closed convex sets of submartingales~\cite{cooman2015:markovergodic}.

One topic that has been studied extensively, is the limit---or stationary---behaviour of imprecise Markov chains, which is very important and useful when looking at practical applications.
Necessary and sufficient conditions for convergence to a unique attractor, expressed in terms of properties of transition operators, were studied by De Cooman and Hermans \cite{cooman2008,hermans2012}: although such convergence seems easier to attain for imprecise Markov chains, its characterisation becomes rather more involved.
Škulj and Hable \cite{skulj2013} characterised unique convergence in terms of coefficients of ergodicity: in addition to providing conditions for convergence, these also measure its rate. 
When there is no unique attractor, imprecise Markov chains display a very different behaviour from precise models. 
The analysis of accessibility between states becomes much more complicated, as does the analysis of the structure of invariant imprecise distributions. 
Škulj \cite{skulj:13b} provided a general classification of such invariant distributions. Also practically useful is the so-called conditional convergence for Markov chains with absorbing states and certain absorption: in the long run the chain will be absorbed into the absorbing state, but, under regularity assumptions, the distribution conditional on the event that the chain has not yet been absorbed converges to a stationary distribution. 
For imprecise Markov chains, Crossman and Škulj \cite{Crossman:2010} proved the existence of a unique imprecise stationary conditional distribution under regularity assumptions. 

More recent work has begun to address other practically useful properties of imprecise discrete-time Markov chains.
Škulj \cite{skulj:2016b} uses coefficients of ergodicity to measure the sensitivity of imprecise Markov chains to perturbations in its parameters. 
A systematic study of expected time to absorption, expected return times, stopping times, hitting  probabilities, and ergodicity was initiated in \cite{troffaes:2013, cooman2015:markovergodic}, but a complete theory is still under development. 
Time reversal and reversible Markov chains they rely on specific properties of transition operators and stationary distributions that turn out to be quite hard to extend to their imprecise versions. 
Although a complete description of reversible processes is not yet possible, a first successful attempt was made by Škulj~\cite{skulj:16}, by studying a special family of reversible imprecise Markov chains that are obtained as random walks on weighted graphs. 

\emph{State of the art for continuous-time imprecise Markov chains}\\[5pt]
The discussion above, as well as the bulk of the literature on imprecise Markov chains, deals mainly with their discrete-time variant: very little is known about how to deal with robustness and imprecision in continuous-time Markov chains. 
Imprecise continuous-time Markov chains can be related---be it rather artificially---to continuous-time Markov decision processes and controlled Markov chains; these notions share the basic idea of multiple possible transition scenarios.
Addressing them using the ideas and efficient techniques of imprecise probabilities has only been attempted very recently \cite{skulj2015:continuous:bounds,Troffaes+GSB-ISIPTA15p,DeBock:2016:iCTMClimit}.
Many basic conceptual, theoretical and computational problems have yet to be solved, and the relationship with the continuous-time versions of imprecise probability trees and closed convex sets if submartingales has yet to be explored.
So far, Škulj \cite{skulj2015:continuous:bounds} has used ideas from the discrete-time approach to introduce imprecise versions of transition rate matrices and establish an appropriate counterpart for Kolmogorov's backward equation, leading to a non-linear differential state evolution equation. 
He proved that it has maximal and minimal solutions that characterise all possible solutions. Finding these boundary solutions numerically is still very much a problem under study, but has huge potential for practical applications.

\vspace{3mm}

\textbf{Describe the objectives of the research.}\\
\textit{Describe the envisaged research and the research hypothesis, why it is important to the field, what impact it could have, whether and how it is specifically unconventional and challenging.}

The main objective of the proposed research project is to make significant progress in the development of the theory of imprecise \emph{continuous-time} \emph{finite-state} Markov chains, using recent developments in imprecise probabilities \cite{augustin2013:itip,troffaes2013:lp} and imprecise probability trees \cite{shafer2001,cooman2007d}, and building on what has already been achieved in the discrete-time case \cite{cooman2008,hermans2012,skulj2013,cooman2015:markovergodic}. In particular, we want to develop the following three lines of research; more detailed information is provided in the Methodology section below.
% \vspace{6pt}
\begin{enumerate}[label=\tiny$\blacksquare$,leftmargin=*,noitemsep]
\item Develop a rigorous martingale-theoretic and measure-theoretic definition for the joint model of a continuous-time imprecise Markov chain, and use this model to generalise a number of basic theoretical properties of precise continuous-time Markov chains.
\item Develop computational methods for imprecise continuous-time Markov chains and conduct a theoretical analysis of their efficiency, with a particular focus on convergence speed (coefficients of ergodicity) and stability (perturbation analysis).
\item Study and characterise the limit behaviour of imprecise continuous-time Markov chains: ``Under which conditions will they converge to a stationary limit distribution?'', ``What properties does such a limit distribution have?'' and ``How can we efficiently compute it?''.
\end{enumerate}
% \vspace{6pt}
Over the past decades, similar lines of research have received plenty of attention for \emph{precise} continuous-time Markov chains, and this has resulted in their successful application in various domains, including control engineering, queueing, artificial intelligence, bio-informatics and survival analysis.
However, for \emph{imprecise} continuous-time Markov chains, these topics are almost completely unexplored.
%Therefore, we believe that this research could lead to a breakthrough in applied imprecise probability: if successful, our results will provide a theoretical basis that allows for the application of imprecise continuous-time Markov chains in the very same domains where precise continuous-time Markov chains are currently being used.
Our research hypothesis is that by developing these three lines of research, it will become technically feasible to successfully apply imprecise continuous-time Markov chains to the same types of problems that are currently solved with their precise versions, and provide them with more robust solutions.

Indeed, the practical advantage of our `imprecise' approach is that it explicitly acknowledges that we may not know the relevant probabilistic models exactly, and that, as a result, our inferences must be robust against such model uncertainty. 
The framework of results and inference methods to be developed will be able to predict the effect of this model uncertainty on specific interesting functions of state variables, for example by providing lower and upper bounds on their expectations. 
The results of such inferences can then be used in various applications and situations, for instance to make safer design choices that are more robust against modelling errors.

The proposed research is unconventional from the perspective of both domains that intersect in this project.
From the point of view of Markov chain theory, it is unconventional because it does away with two traditional assumptions: the Markov condition is replaced by a weaker version, and the probabilities that make up the model do not have to be specified exactly. 
And from the imprecise probability perspective, this project is unconventional because time is taken to be continuous, whereas past research on imprecise stochastic processes has focussed almost unanimously on discrete-time problems.

Due to the unconventional nature of the proposed research, and because the topic is largely unexplored, the project will clearly be challenging. 
However, given the extensive experience of both partners in studying various aspects of imprecise \emph{discrete-time} stochastic processes~\cite{cooman2007d,cooman2008,hermans2012,cooman2015:markovergodic}, we are convinced we have the necessary expertise to successfully tackle similar research problems in continuous time.


\vspace{7pt}

\textbf{Describe the methodology of your research.}\\
\textit{Be as detailed as necessary for a clear understanding of what you propose.
Describe the different envisaged steps in your research, including intermediate goals. Indicate how you will handle unforeseen circumstances, intermediate results and risks.
Show where the proposed methodology is according to the state of the art and where it is novel.
Enclose risks that might endanger reaching project objectives and the contingency plans to be put in place should risk occur.}

As explained in the objectives section, we intend to pursue three lines of research, each of which has been very successful for precise continuous-time Markov chains, but none of which has so far seen significant progress in the imprecise case; a detailed plan for each is outlined below. 
Each research lines will initially be carried out by a dedicated full-time PhD student, under the close supervision of the (co-)supervisors and a part-time senior researcher. 
As time and research progresses, we expect these research lines to benefit from each other, leading to intensive collaborations between the respective students (and their supervisors). Concrete topics for collaboration are discussed within the following detailed plans.

\emph{Joint models and basic properties} {\bf (Research Line 1)}.\\[3pt]
A first important goal of this project is to develop a rigorous martingale-theoretic and measure-theoretic definitions for the joint models of a continuous-time imprecise Markov chains, and to use these models to generalise a number of basic theoretical properties of precise continuous-time Markov chains. 
%This research will be conducted by a single student, under the close supervision of the IP unit of the Data Science Lab, and can be divided into the following six work packages.
This research can be divided into the following three work packages.
% \vspace{6pt}
\begin{enumerate}[label=\tiny$\blacksquare$,leftmargin=*,noitemsep]
% \item[\tiny$\blacksquare$]
% The first step in this line of research will be for the student to \emph{study the relevant existing literature on imprecise probabilities and imprecise stochastic processes} as described above, and to familiarise herself thoroughly with measure-theoretic and martingale-theoretic approaches to probability {\bf(Work Package 1a)}.
\item The first step will consist in \emph{investigating how to define an imprecise probability tree, and in particular an imprecise Markov chain, in continuous time}: what are the possible paths and nodes, how can we define local imprecise probability models attached to these nodes, and how do we formulate a Markov condition for them?
In addition, it must be investigated how to define the submartingales associated with such a continuous-time imprecise probability tree, and whether and how this definition simplifies when we impose the Markov condition {\bf(Work Package~1a)}.
To achieve this, we can draw inspiration from earlier game-theoretic probability work done by Vovk in characterising the (precise) probability tree associated with continuous-time Brownian motion \cite{vovk2008:brownian,vovk2012:emergence:of:probability}, a special case of a Markov chain.
% with fully independent local models.
%Indeed, much of the this line of research can be seen to consist in generalising and extending his ideas from Brownian motion to Markov chains.
%Therefore, the fact that solid and very interesting results can be obtained for Brownian motion---which is a special case of a Markov chain with fully independent local models---strongly indicates that the present line of research is indeed feasible, though not trivial.
\item In a second step, we use the results of the first to obtain an expression for the joint probability model---a so-called lower expectation operator on the sample space of possible paths---of an imprecise continuous-time Markov chain that is amenable to further exploration {\bf(Work Package 1b)}.
We also investigate whether this lower expectation operator can be expressed as a lower envelope of expectation operators associated with more traditional, measure-theoretic stochastic processes.
%It may also be interesting, if rapid progress can be made and time allows it, to find out whether---as in the discrete-time case \cite{cooman2015:isipta:markov,cooman2015:markovergodic}---such an expression for the joint will allow for a formulation of ergodic theorems for continuous-time imprecise Markov chains.
\item The final step consists in studying the mathematical properties of the continuous-time imprecise Markov model {\bf(Work Package 1c)}.
As a first special case, we intend to check whether it allows for a derivation from first principles of a non-linear variant of the Chapman--Kolmogorov equations, similar to what \v{S}kulj \cite{skulj2015:continuous:bounds} found using more {\itshape ad hoc} arguments.
That something similar is possible---and fairly straightforward---in the discrete-time case \cite{cooman2008}, should serve as an indication that this is indeed feasible. 
A second special case will be to define stopping times, and to study their basic properties.
\end{enumerate}
Since the Chapman-Kolmogorov equations and stopping times lie at the basis of many computational methods for (precise) continuous-time Markov chains, we expect Work Package~1c to yield results that are useful to advance the second---computational---line of research that we are about to introduce (especially Work Package~2b). In order to take advantage of this overlap, we will provide the respective researchers with the opportunity to collaborate on these topics.

\emph{Computational methods} {\bf (Research Line 2)}.\\[3pt]
While precise Markov chain theory often relies heavily on linear algebra, this is no longer true for imprecise Markov chains, where we need to resort to linear programming techniques. 
Consequently, in the majority of cases, solutions will have to be obtained in numerical form.
Computational methods are therefore important, and must be developed alongside with theoretical concepts. 
The second general goal of this project is therefore to develop and improve numerical methods. This line of research can be divided into the following four work packages.

%This is especially true for the continuous time case, because in principle we would have infinite number of time steps where a linear programming problems would have to be solved. 
%Instead of that we have to do some kind of discretisation. 
%But this approach generates errors, that have to be managed. 
%Moreover, the existing methods capable of providing bounds of distributions at given finite time points~\cite{skulj2015} fail when it comes to infinite time intervals. 
%Actually, they start behaving poorly when the time intervals become larger. 
%The reason is that they are based on the theory of differential equations whose bounds grow exponentially in time.
% \vspace{6pt}
\begin{enumerate}[label=\tiny$\blacksquare$,leftmargin=*,noitemsep]
\item The first step will be to define and compute coefficients of ergodicity (and other contraction measures) for imprecise continuous-time Markov chains {\bf(Work Package 2a)}. 
By analogy with the discrete-time case~\cite{skulj2013}, we expect it will be difficult to evaluate the theoretical definitions directly, so indirect computational methods will need to be developed. 
To do so, we will need to be able to efficiently compute distances between coherent lower previsions that are only partially specified. 
Known numerical examples suggest that this problem is challenging but solvable. 
%To solve this problem we will use methods of convex analysis and the theory of perturbations of linear programming problems.
\item The next task {\bf(Work Package 2b)} is to develop methods for calculating tight bounds on expectations of functions of the states of an imprecise Markov chain. 
First, we intend to extend existing methods for functions that depend on the state at a single time in the future~\cite{skulj2015:continuous:bounds}, which fail when the time interval grows large. 
To overcome this, we intend to exploit the contracting nature of Markov transition operators, using the coefficients of ergodicity of Work Package 2a. 
Later, we will also try and compute expectations for more general functions, including important stopping times---the expected time to absorption and expected return times---and probabilities of general events---absorption probabilities.
\item The third task {\bf(Work Package 2c)} is error estimation.
Regardless of the method used, numerical errors seem inevitable in continuous-time models, where only a finite number of optimisation steps are feasible, whereas in theory, an optimisation would actually be needed in each intermediate time point. 
This gives rise to numerical errors, which are in every step transferred to the next steps as perturbations. 
Existing error estimates work well for short time intervals, but tend to significantly overestimate the errors for larger time intervals, as has been observed experimentally in simple problems.
\item An important part of the error estimation will be a perturbation analysis, which is also of independent interest and will therefore be explored as a separate research task {\bf(Work Package 2d)}.
We will analyse how perturbations of its parameters affect the distance between an imprecise continuous-time Markov chain and its perturbed version.
Such a perturbation analysis has been successfully applied to imprecise discrete-time Markov chains; we expect it to be equally successful in continuous time.
\end{enumerate}
%%Develop computational methods for imprecise continuous-time Markov chains and conduct a theoretical analysis of the efficiency of these methods, with a particular focus on aspects such as convergence speed (coefficients of ergodicity) and stability (perturbation analysis).
%
%(feel free to change the order of the sub-workpackages as you like...)
%
%{\bf (Work Package 2a)}: coefficients of ergodicity (and perhaps other speeds of convergence)
%
%*** add text ***
%
%*** ergodicity has been characterised in Jasper's paper, but we don't know yet the speed of convergence... ***
%
%{\bf (Work Package 2b)}: computing lower expectations on a single time point, conditional on initial state, for large values of $t$ (for which current methods are now still exponential)
%
%*** add text ***
%
%{\color{Gray}
%old text that may be useful:
%
%The steps discussed above are purely theoretical, and consist of developing a formal framework in which to study continuous-time imprecise Markov chains.
%In order to be able to use these models in practice, this formal framework will need to be complemented with efficient computational methods.
%Step five therefore consists in developing efficient algorithms for calculating lower and upper expectation bounds in a number of interesting and useful special cases
%}
%
%
%analysis of the limit behaviour of imprecise continuous time Markov processes
%
%
%
%{\bf (Work Package 2c)}: perturbation analysis study
%
%*** add text ***
%
%\emph{The dependence of the imprecise continuous time Markov chains on the parameters.} 
%
%{\bf (Work Package 2d)}: distances between partially specified lower expectations and how they can be used for part a,b and c.
%
%*** add text ***
%
%*** Here I mean the problem of calculating the distances between the natural extensions of lower previsions, based on the differences obtained on their domains. That is what is $\max_{0\le f \le 1} |\underline E(f)-\underline E'(f)|$ if $ |\underline E(h)-\underline E'(h)|$ is given for every $h \in \mathcal H$. 
% 
% Another related problem is what is the maximal possible distance between two coherent lower previsions that coincide on a given set of gambles. 
% 
% The above results are needed if we want to calculate coefficients of ergodicity for $\overline T^n$ or $T_t$ in continuous time. 
%
% An indispensable tool for evaluating convergence in terms of how close are the results to the true values is the metric structure of the imprecise probability models. In this area a big gap exists between the theory and practice. We will reduce this gap by providing efficient methods for calculating the distances between imprecise probability models. The models of imprecise probabilities are usually evaluated as solutions of linear programming problems or, especially in the theory of imprecise Markov chains as sequences of several linear programming problems. Therefore a lot of computational effort is needed to obtain values of parameters, and consequently there is usually only a limited numbers of parameters at disposal. We will therefore look for the methods that will enable estimating the distances between imprecise probability models based on their partial identification. To solve this problem we will use methods of convex analysis and the theory of perturbations of linear programming problems.

As discussed, this computation-oriented research will benefit from the theoretical results in Research Line 1, and the respective researchers will be expected to collaborate. Similarly, the computational methods that we develop here (Work Package 2b and 2c) will benefit the study of limit distributions that we intend to conduct for Research Line 3 (Work Package 3b), because this study will involve computing limit distributions numerically; here too, the respective researchers will be expected to collaborate.

\emph{Limit behaviour and stationary distributions} {\bf (Research Line 3)}.\\[3pt]
Limit behaviour is a crucial topic in Markov chain theory, and if a unique stationary distribution exists, it is often a central object of interest. 
In the precise case, linear algebra provides powerful tools for the analysis of limit behaviour, but no such methods are available for imprecise Markov chains. 
For imprecise discrete-time Markov chains, alternative methods have been developed~\cite{cooman2008}, but in continuous time, little is known. 
Therefore, our third main goal is to study and characterise the limit behaviour of imprecise continuous-time Markov chains. 
This line of research can be divided into three work packages:
% \vspace{6pt}
\begin{enumerate}[label=\tiny$\blacksquare$,leftmargin=*,noitemsep]
\item The first task will be to find out under which conditions an imprecise continuous-time Markov chain converges to a limit distribution {\bf (Work Package 3a)}. 
When such a limit distribution exists, we will also characterise the conditions under which it depends on the initial state of the chain and the conditions under which it is stationary. 
In discrete time, such a study has already been conducted successfully~\cite{hermans2012,cooman2008}. 
In continuous time however, we are only aware of some partial answers \cite{DeBock:2016:iCTMClimit}.
\item Once we have established the existence of a limit distribution, the next task will be to develop methods that allow us to compute it~{\bf (Work Package 3b)}. 
In the precise case, this is relatively easy, as it requires solving a linear system of equations. 
In the imprecise case, this system becomes non-linear, and solving it is far more difficult. 
We envision two approaches. 
The first option is to try and solve this system directly. 
We expect this to be feasible only in special cases. 
Whenever it is not, we will resort to the second option, which is to compute the limit distribution more naively: to actually approach the limit up to some desired accuracy. 
Since this second option will clearly benefit from the results in Work Packages 2b and 2c, we intend for the respective researchers to collaborate on these topics.
\item The last task  will be to analyse the structure of limit distributions: the types of classes the states belong to {\bf (Work Package 3c)}.
We expect this to be challenging because, due to their generality, imprecise continuous-time Markov chains will allow for a far richer range of classes of states and transitions between them than in the precise case. 
Our aim is to investigate the induced accessibility relations and their influence on convergence, especially when it is not unique, and to characterise and classify the invariant distributions. 
As a first step, we will extend the methodology in~\cite{skulj:13b}---a partial classification of invariant sets for discrete-time imprecise Markov chains---to continuous time. 
\end{enumerate}

\emph{Contingency plans} {\bf (Work Packages 4 \& 5)}.\\[3pt]
Although we expect to finish each work package, we might of course run into problems: some computations might be inherently intractable, or there might be theoretical questions we are unable to answer.
Even so, this would not endanger the aim of our project: to make significant progress in the field of continuous-time Markov chains.
Indeed, since this field is still essentially wide open, and because both project partners have been instrumental in developing the discrete-time version, it should not be a problem to make significant progress in each of the three proposed research lines.
Nevertheless, should we need to absorb major setbacks, the following additional research topics can serve as contingency plans.
They also serve as extra topics should the intended research proceed faster than expected.

First, as is clear from the discussion above, the proposed research does not work towards a specific application, nor does it try to build an ad-hoc model for some specific problem.
Rather, the aim is to develop a general theory of continuous-time Markov chains.
Nevertheless, our results can be used to study and solve a variety of practical problems.
Since real applications are an ideal testing ground for new theory, it would be nice to tackle such applications {\bf (Work Package 4)}; the choice of application will depend on the level of complexity that the developed framework will by then be able to deal with.

A second extra research topic is time reversibility.
This concept is very important for (precise) stochastic processes, but it has not yet been systematically studied in the framework of imprecision; to the best of our knowledge, the preliminary---and so far unpublished---results~\cite{skulj:16} are the only exception.
Therefore, a systematic analysis of imprecise continuous-time reversible Markov chains {\bf (Work Package 5)} would definitely be an interesting research topic to pursue.
%The first task is a formal approach to time reversal for processed based on imprecise probabilities
% This is a theoretically challenging task, since the concept applies techniques, such as conditioning and Bayes rule, whose applications to imprecise probability models substantially differ from their use in the classical models. We will examine the possibilities to model time reversal in terms of the models of imprecise probabilities and establish the relation between continuous random walks on weighted graphs, extending the methodology used in  \cite{skulj:16} for the discrete time case. A special emphasis will be on the computational issues. 


%The work will be divided into the following ??? work packages.
%
%\vspace{6pt}
%\begin{itemise}
%	\item[\tiny$\blacksquare$] 
%	Within  {\bf(Work Package 4a)} conceptual properties of time reversal will be investigated. We will answer the following questions. Is a time reversed imprecise Markov chain still Markovian if it is imprecise. Can a time reversed imprecise Markov chain be analysed in terms of convex sets of probabilities. These questions will be addressed using the theory developed within the first line of research. 
%	
%	\item[\tiny$\blacksquare$]
%	Partially independently from time reversal for general imprecise Markov chains we will study reversible Markov chains, which are those, whose stationary joint probability distributions do not change if the time is reversed within {\bf(Work Package 4b)}. We expect that families of reversible imprecise Markov chains can be constructed without necessarily having all answers regarding time reversal solved. As a starting point we will generalise the model of random walks on weighted graphs  proposed in \cite{skulj:16} to the continuous time framework.
%\end{itemise}
%We must first answer the question whether a time reversed imprecise Markov chain is still Markovian if it is imprecise, and what properties does it have. The next task is to identify reversible imprecise Markov chains, which are those, whose properties do not change if they are reversed. Along with the theoretical questions we will explore practical/computational aspects of the model proposed in \cite{skulj:16} extended to the continuous time framework. 
%
%Time reversal for imprecise Markov chains will be examined. 
%In particular, we will find out under what conditions is a reverse process Markov, and specifically, whether it fits into the framework of imprecise Markov chains examined. 
%In particular, we will examine the continuous time version of the model of random walks on graphs.  
%(tentative title, feel free to change)





% ## Addedd by Damjan

%The first step in this project will be to \emph{develop robust versions of the basic building blocks of queueing models, that is, imprecise arrival and service-time processes} {\bf(Work Package 1)}.
%Initially, given their popularity and importance, we will focus on developing imprecise versions of Poisson processes and exponential distributions, with increasing degrees of robustness.
%A first level of robustness will be added by allowing the parameter of Poisson processes and exponential distributions to vary within intervals.
%Next, additional layers of robustness will be added by allowing this parameter to be time-dependent and eventually even history-dependent.
%We intend to achieve this in a very rigorous way, by dropping the independence axioms that are traditionally used to define Poisson processes and exponential distributions, and replacing them with increasingly weaker axioms.
%Each additional layer of robustness will result in a more flexible and realistic uncertainty model, at the cost of increased theoretical and computational complexity.
%However, given that the Data Science Lab has already successfully tackled this kind of issues for discrete-time robust stochastic processes~\cite{hermansITIP}, we expect to be able to obtain similar successes in this continuous-time case. In a later stage, we will try to robustify other continuous-time distributions and arrival processes, such as Erlang distributions and Markovian arrival processes.

% % Damjan 



% # Damjan

%The second step in this project consists in using the imprecise arrival and service-time processes of the first step to \emph{develop robust queueing models} {\bf(Work Package 2)}.
%Basically, the idea is to consider existing queueing models, and to replace their arrival and service-time processes by our imprecise versions.
%We intend to start by developing a robust version of the $M/M/1$ queue, where customers arrive according to an imprecise Poisson process, one server serves the waiting customers in an FCFS order, and service times have an imprecise exponential distribution.
%In the precise case, the number of customers in the queue can then be described by a continuous-time birth-death chain.
%Similarly, we would like to show that in the imprecise case, the number of customers can be described by a continuous-time imprecise birth-death chain, which is a simple type of continuous-time imprecise Markov chain~\cite{Skulj2015}.
%Similarly, for more complex queueing models, we intend to show that the number of customers can be described by more complex imprecise Markov chains.
%
%Clearly, step two will require us to \emph{establish a rigorous theoretical basis for continuous-time imprecise Markov chains}, and we regard this as a third part of this research {\bf(Work Package 3)}, which we intend to initiate before---and then continue in parallel with---the second step. Some preliminary research has already been conducted on this topic~\cite{Skulj2015}, including some recent unpublished work by the imprecise probability subunit of the Data Science Lab, but it is fair to say that there are still many unsolved basic theoretical and computational problems left. For example, past research on imprecise continous-time Markov chains only considered finite state spaces, whereas basic queueing theory often considers countably many states. %\todo{*** what do you mean by that? ***} *** references ***
%
%The three steps that were discussed above are purely theoretical, and consist of developing a formal framework in which to study imprecise arrival and service-time processes, imprecise Markov chains, and ways of combining these to construct robust queueing models.
%In order to be able to use these models in practice, this formal framework will need to be complemented with efficient computational methods.
%In particular, it will be necessary to develop methods that are able to provide us with exact bounds on performance measures---such as the average occupancy or delay---that reflect the extent to which the imprecision in the input affects the output.
%Indeed, once we have these bounds, they can then be used to make robust a priori design choices while dimensioning a queueing system, such as determining the needed buffer space or the desired number of servers, in order to achieve the desired performance.
%Similarly, for systems whose number of servers can vary over time, these bounds can be used to design robust dynamic server policies.
%Given the importance of these bounds, a fourth step in this project will be to \emph{extend the existing computational methods for continuous-time imprecise Markov chains}~\cite{Skulj2015,Troffaes+GSB-ISIPTA15p}, and to \emph{use these methods to efficiently compute exact bounds on performance measures} {\bf(Work Package 4)}.
%
%Although we expect to reach the majority of the above goals, it might of course happen that we do not. For example, some types of performance measures might turn out to be inherently intractable to compute with, or we might come across stochastic processes for which adding three layers of robustness (allowing their parameters to be interval-valued, time- and history dependent) results in a model that is simply too complex to work with. However, even if we were to come across such problems, this will not endanger the global aim of the first part of this project, which is to develop robust continuous-time queueing models. Indeed, there are various types of stochastic processes, increasingly complex ways of adding robustness, and many performance measures of interest. Adding a modest amount of robustness to simple queueing models will definitely be possible, and this alone suffices to continue with the rest of this project. However, of course, our intention is to go much further: ideally, we would like to develop a general framework for working with robust continuous-time queueing models.
%%In any case, regardless on how general our study ends up being, we definitely expect to be able to develop robust continuous-time queueing models, and this will allow us to move to a next step.
%
%As should have become clear from the explanation above, our proposed research does not work towards a specific application, neither does it try to build an ad hoc model for some specific problem.
%Instead, our aim is to build a general framework for robust continuous-time queueing theory.
%Nevertheless, if we succeed in developing such a general framework, then of course, it can be used to study and solve a variety of problems, in different types of queueing applications. In order to demonstrate this, we intend to \emph{use our framework to analyse one or two specific queueing applications} {\bf(Work Package 5)}.
%Our choice of applications will depend on the level of complexity that our framework is able to deal with, but since robustness is an essential feature of our approach, we intend to focus on problems where this feature is important, that is, applications where the exact values of the parameters of the model are uncertain (e.g. telecommunication) and/or non-stationary (e.g. call centres \cite{Defraeye20164}).
%
%The purpose of this applied side of the project is only to test and validate the usefulness of our theory. Therefore, tackling this applied part will not require us to learn imprecise stochastic processes from data (e.g.\ Internet traces). For the purposes of this project, we can simply start from precise stochastic processes---learnt by means of existing methods, that is, by proposing a model that is deemed adequate and then applying estimation theory (see e.g.\ \cite{Breuer2002}) to estimate its parameters---and can then add increasing amounts of imprecision to these precise stochastic processes in order to study the robustness of the resulting outputs. %The initial precise model can then be learnt by means of existing methods, that is, by proposing a model that is deemed adequate and then applying estimation theory (see e.g.\ \cite{Breuer2002}) to estimate its parameters.
%However, in a later phase, it would of course be better to \emph{develop methods for learning imprecise continuous-time stochastic processes directly from data} {\bf(Work Package 6)}. At this point, we envision that this last topic will only be marginally treated in the project. However, if necessary, it does provide us with a contingency plan that is of independent interest.

\vspace{7pt}

\textbf{Provide a work plan, i.e. the different work packages and detailed timetable.}\\
\textit{Describe the different work packages (WP) the proposed research work will be divided in.
Indicate for each WP the time that it is expected to take.
You might use a table or another type of scheme to clarify the work plan. Clearly indicate the contribution of each project partner, taking into account the complementary expertise of the project partners.}

% {\bf\color{blue} (WE WAIT WITH THIS PART FOR NOW...)}

At the universities of Ghent and Ljubljana, a PhD fellowship, and therefore also the work on the proposed research, is usually spread over a period of four years.
We have indicated these four years in the table below, and have divided every year into four periods of three months (i.e.~quarters).
The work on the research was divided in work packages in the previous section.
For each work package, the table provides a short description and indicates the quarters during which we expect the researchers to be working on it.
We mark the boxes with a G indicating that the research on the topic in the specific quarter will be led by the Ghent group; an L indicates that the research will be led by the Ljubljana group. 
After the initial research on a topic has been carried out, we plan to organise the discussion of the results, followed by joint work to finalise them. 
B indicates the quarters where the work on the topics will be carried out by both groups. 

After the work on a specific topic is finished, the results will be disseminated at research conferences and published in scientific journals. 
The last two quarters are reserved for finishing the PhD theses. 

%We tick the box of a quarter with an x to indicate almost certainties; these are the quarters during which we expect her to definitely spend time on the indicated work package.
%We use o's to tick boxes for which we are not yet certain that she will need and/or use them.
%These o-quarters are used to absorb setbacks or to allow for some flexibility, and indicate the maximum amount of time that we intend the student to spend on a given work package.
%Of course, a good deal of the time in the last two quarters of year four will also be spent writing the PhD thesis, but we have not marked this in the table.

\vspace{7pt}
\begin{center}
\resizebox{1\textwidth}{!}{%
\begin{tabular}{r|l|c|c|c|c|c|c|c|c|c|c|c|c|c|c|c|c}
& \emph{Short description} & \multicolumn{4}{c|}{\textbf{\nth{1} year}}   & \multicolumn{4}{c|}{\textbf{\nth{2} year}} & \multicolumn{4}{c|}{\textbf{\nth{3} year}} & \multicolumn{4}{c}{\textbf{\nth{4} year}} 
\\ 
\hline
%\textbf{RL1} 
& \textbf{Joint models and properties}
&   \multicolumn{16}{c}{}     
\\ \hline
\textbf{WP1a} & Probability tree model  
&  &  &  &  &  &  &   &   &   &   &   &   &   &   &   &    
\\ \hline
\textbf{WP1b} & Expression for the joint          
&   &  &  &  &  &  &   &   &   &   &   &   &   &   &   &    
\\ \hline
\textbf{WP1c} & Mathematical properties         
& G  & G  & G  & B & B & B & B & B & B &   &   &   &   &   &   &    
\\ \hline
%\textbf{RL2} 
& \textbf{Computational methods}
&   \multicolumn{16}{c}{}     
\\ \hline
\textbf{WP2a} & Coefficients of ergodicity 
&   &   &   &   &   & L  & L & L & B & B & B &  &  &  &   &    
\\ \hline
\textbf{WP2b} & Compute thight bounds             
& L  & L  & L  & L  & L  & L  &  B &  B & B  &  &  &  &  &  &  &  
\\ \hline
\textbf{WP2c} & Error estimation    
& L  & L  & L  & B  & B  &  B &   &   &   &   &  &  &  &  &  & 
\\ \hline
\textbf{WP2d} & Perturbation analysis     
&   &   &   &   &   &   &   &   & L  & L  & L & L & B & B &  & 
\\ 
\hline
%\textbf{RL3} 
& \textbf{Limit behaviour}
&   \multicolumn{16}{c}{}     
\\ \hline
\textbf{WP3a} & Existence of limit distributions
&  &  &  & G & G & G & G  & B  & B  &   &   &   &   &   &   &    
\\ \hline
\textbf{WP3b} & Compute limit distributions      
&   &  &  &  & G & G & G  & G  & B  & B  & B  & B  &   &   &   &    
\\ \hline
\textbf{WP3c} & Characterise limit distributions   
&   &   &   &  &  &  &  & L & L & L  &  B &  B & B  & B  &   &  
\\ \hline 
& \textbf{Contingency plans}
&   \multicolumn{16}{c}{}     
\\ \hline
\textbf{WP4} & Applications 
&   &   &   &   & G  & G  & G & G & B & B & B & B & B & B &   &    
\\ \hline
\textbf{WP5} & Time reversal         
&   &   &   &   & L  &  L &  L &  L & B  & B & B & B & B & B &  &   
\end{tabular}}
\vspace{10pt}
\end{center}
% As explained in our methodology, the work on (WP2) is initiated after the work on (WP3) because it relies on some of its results. Furthermore, as also explained before, (WP6) is currently not at the core of the proposed research, but can serve as a contingency plan if necessary.


%As explained in our methodology, the work on (WP2) is initiated after the work on (WP3) because it relies on some of its results. Furthermore, as also explained before, (WP6) is currently not at the core of the proposed research, but can serve as a contingency plan if necessary.

% Alternative:
% \begin{table}[H]
%     \caption{Summary of work packages}
%     \centering
%     \begin{tabular}{c l}
%         \toprule
%                      Work package & Very short description \\
%         \midrule
%        (WP1)    & Imprecise arrival and service-time processes \\
%        (WP2)    & Development of robust queueing models \\
%        (WP3)    & Continuous-time imprecise Markov chains \\
%        (WP4)    & Computing bounds on performance measures \\
%        (WP5)    & Learning robust queueing models from data \\
%        (WP6)    & Applications of robust queueing models \\
%        \bottomrule
%     \end{tabular}
%     \begin{tabular}{l c cccc c cccc c cccc c cccc}
%         \toprule
%         & \hspace{.1em} & \multicolumn{4}{c}{\nth{1} year}   & \hspace{.5em}    & \multicolumn{4}{c}{\nth{2} year}      & \hspace{.5em} & \multicolumn{4}{c}{\nth{3} year} & \hspace{.5em} & \multicolumn{4}{c}{\nth{4} year} \\
%         \midrule
%         WP1    & & x & x & x & x &   & x & x &   &   &   &   &   &   &   &   &   &   &   &    \\
%         WP2    & &   &   &   &   &   & x & x & x & x &   & x & x & x &   &   &   &   &   &    \\
%         WP3    & &   &   &   &   &   &   &   &   &   &   &   &   & x & x &   & x & x & x & x   \\
%         WP4    & &   &   &   &   &   &   &   &   &   &   &   &   & x & x &   & x & x & x & x   \\
%         WP5    & &   &   &   &   &   &   &   &   &   &   &   &   & x & x &   & x & x & x & x   \\
%         WP6    & &   &   &   &   &   &   &   &   &   &   &   &   & x & x &   & x & x & x & x   \\
%         \bottomrule
%     \end{tabular}
% \end{table}

%\vspace{3pt}

\bibliographystyle{plain}
\renewcommand\refname{\normalsize Enumerate the bibliographical references that are relevant for your research proposal.}
\bibliography{bibliography,general}

\vspace{5mm}

\textbf{Indicate below whether you think the results of the proposed research will be suitable to be communicated to a non expert audience and how you would undertake such communication.}\\
\textit{FWO encourages its fellows to disseminate the results of their research widely, and valorize them where possible.}

Markov chains are definitely suitable for communication to a non-expert audience. 
Their basics are easy to explain to an audience with limited maths. 
Also, they can be used to describe many practical problems, ranging from basic strategies for playing certain games to demographic predictions, planning of workforce recruitment and promotion strategies. 
For example, UGhent research on queueing theory---which uses Markov chains extensively---has recently been cleverly applied to waiting in line at a festival bar, and its findings were easily picked up by several national media.
The addition of imprecision to such problems only increases their potential for communication to a non expert audience, because it allows us to include aspects such as safety, reliability and robustness, all of which are highly relevant to society.

In addition to purely educative or demonstrative purposes, Markov chains can also be used to develop relatively simple web-based tools that can serve to support business processes. 
The Slovenian group has been involved in the successful development of such a web-based platform for prediction and strategy testing for the structure of Slovenian armed forces. 
Such platforms would definitely benefit from models that allow for imprecision as well.
\end{document}