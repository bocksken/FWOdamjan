\documentclass[10pt,a4paper,sans]{moderncv}
  %\moderncvtheme[roman,blue]{classic}
  \moderncvstyle{classic}                             % style options are 'casual' (default), 'classic', 'oldstyle' and 'banking'
\moderncvcolor{blue} 
% \usepackage{helvet}
\usepackage{xcolor}
\usepackage[official]{eurosym}
%\usepackage[utf8]{inputenc}
\usepackage[scale=0.75]{geometry}
\setlength{\hintscolumnwidth}{1.1cm}						% if you want to change the width of the column with the dates
%\AtBeginDocument{\setlength{\maketitlenamewidth}{6cm}}  % only for the classic theme, if you want to change the width of your name placeholder (to leave more space for your address details

% personal data
\firstname{Gert}
\familyname{de Cooman}
%\title{March 2016}  % optional, remove the line if not wanted

\mobile{+32496832628}  % optional, remove the line if not wanted
%\phone{+1-412-268-9669} % optional, remove the line if not wanted
%\fax{+1-412-268-1440}   % optional, remove the line if not wanted
\email{gert.decooman@UGent.be}  % optional, remove the line if not wanted
\address{Ghent University}{IDLab}  % optional, remove the line if not wanted
\extrainfo{\href{http://users.ugent.be/~gdcooma}{users.ugent.be/$\sim$gdcooma}} % optional, remove the line if not wanted
\photo[6\baselineskip]{GertdeCooman}  % optional, remove the line if not wanted
% \quote{Some quote (optional)} % optional, remove the line if not wanted
%\nopagenumbers{} % uncomment to suppress automatic page numbering for CVs longer than one page


%----------------------------------------------------------------------------------
%            content
%----------------------------------------------------------------------------------
\begin{document}
%\renewcommand*{\firstnamefont}{\fontsize{30}{32}\sffamily\mdseries\upshape}
\renewcommand*{\cventry}[6]{%
  \cvline{#1}{%
    {\scshape#2}%
    \ifx#3\else{, {\slshape#3}}\fi%
    \ifx#4\else{, #4}\fi%
    \ifx#5\else{, #5}\fi%
    .%
    \ifx#6\else{\newline{}\begin{minipage}[t]{\linewidth}\small#6\end{minipage}}\fi
}}%
\colorlet{titlecolor}{darkgray}
\colorlet{addresscolor}{black!100}
\colorlet{sectionrectanglecolor}{blue!10}
\colorlet{sectiontitlecolor}{blue!100}
\colorlet{subsectioncolor}{blue!80}


\maketitle

\vspace{-30pt}

\section{Short CV}
\vspace{4pt}

Gert de Cooman (born 30 September 1964) is Full Professor of Uncertainty Modelling and Systems Science at Ghent University (Belgium), IDLab, and Honorary Visiting Professor at Durham University (UK), Department of Mathematical Sciences. He has been a PhD Fellow, Postdoctoral Fellow, and Research Associate of the FWO, and a Visiting Research Scholar at Binghamton University (SUNY), during 1997. He obtained his PhD in Engineering from Ghent University in May 1993.

\hspace{10pt}He is the author of 70 journal and 72 conference papers, several edited books, and a research monograph, {\itshape Lower Previsions}, co-authored with Matthias Troffaes and published by Wiley (433 pages). 
He is or has been the promotor of the doctoral research of 8 PhD students (currently guiding 3 PhD students) at Ghent University, and of 25 master’s theses in Engineering and in Mathematics at Ghent University.
He was invited to serve in more than 30 doctoral examination committees at Ghent University and abroad.
He has been involved in the organisation of more than 25 conferences, workshops and summer schools, served as a member of the programme committee and as a reviewer for more than 30 conferences.
He has also been invited to give more than 50 plenary lectures at conferences, university departments and research institutes (amongst which LANL, Santa Fe Institute, Durham University, NASA/NIA, University of Tokyo, Sun Yat-Sen University, University of Saskatchewan).
In 2006, he was elected Grey College Alan Richards Fellow in Mathematics (Easter term) at Durham University.

\hspace{10pt}Gert de Cooman has done extensive research in the field of imprecise probabilities, where he is considered as a leading figure, and one of the pioneers.
He has contributed to (and is interested in) such various topics as: sets of desirable gambles and choice functions as foundations for uncertain reasoning and decision making, incorporating possibility theory into the theory of imprecise probabilities, belief change with general information states, missing data, symmetry and exchangeability, laws of large numbers and ergodic theorems for imprecise probability models, predictive inference, imprecise (hidden) Markov models, game-theoretic probability and stochastic processes with imprecise probability, in particular Markov chains with imprecise transition probabilities, and efficient algorithms for making inferences in credal nets under epistemic irrelevance.
He is a founding member of SIPTA, the International Society for Imprecise Probability:
Theories and Applications (an international statistical society whose aim is to promote research in, and applications of imprecise probabilities), and served as its President between 2002 and 2007. He has been also been a member of its Executive Committee, serving in various roles, between 2007 and 2015.

\vspace{2pt}
\section{5 key peer reviewed publications}
\vspace{5pt}
\small
\cvline{2012}{Gert de Cooman and Erik Quaeghebeur. Exchangeability and sets of desirable gambles. {\itshape International Journal of Approximate Reasoning} 53:363–-395. [IF 2014: 2.451; CIT: 11]}
\cvline{2008}{Gert de Cooman and Filip Hermans. Imprecise probability trees: Bridging two theories of imprecise probability. {\itshape Artificial Intelligence} 172:1400--1427. [IF 2014: 3.371; CIT: 20]}
\cvline{2005}{Gert de Cooman. A behavioural model for vague probability assessments. {\itshape Fuzzy Sets and Systems} 154:305--358. With discussion: papers by Seraf\'{\i}n Moral, Lev Utkin, Romano Scozzafava and Lotfi Zadeh. [IF 2014: 1.986; CIT: 39]}
\cvline{1999}{Gert de Cooman and Dirk Aeyels. Supremum preserving upper probabilities, {\itshape Information Sciences}, 118:173--212. [IF 2014: 4.038; CIT: 72]}
\cvline{1997}{Gert de Cooman. Possibility theory I: the measure- and integral-theoretic groundwork. {\itshape International Journal of General Systems}, 25:291--323. [IF 2014: 1.637; CIT: 139]}
\end{document}
